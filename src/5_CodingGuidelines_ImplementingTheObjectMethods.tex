\section{Implementing the Object Methods}
In Java all \textit{classes} are somehow extending the class \lstinline|java.lang.Object|\autocite{ORACLE_DOC_OBJECT_CLASS}, and therefore, they inherit several methods from it. Four of these methods\footnote{In fact, it is five methods, but the method \lstinline|finalize()|\autocite{ORACLE_DOC_OBJECT_CLASS:finalize} is deprecated and should not be used anymore. Refer to chapter \tqfullvref{sec:Finalisation} for more details on this topic.} can be overridden to adjust the behaviour of your class to your needs:
\begin{itemize}[nosep]
\item{\lstinline|clone()|\autocite{ORACLE_DOC_OBJECT_CLASS:clone}}
\item{\lstinline|equals()|\autocite{ORACLE_DOC_OBJECT_CLASS:equals}}
\item{\lstinline|hashCode()|\autocite{ORACLE_DOC_OBJECT_CLASS:hashCode}}
\item{\lstinline|toString()|\autocite{ORACLE_DOC_OBJECT_CLASS:toString}}
\end{itemize}

The chapters below will provide some guidelines on how to code new implementations for these methods.

\subsection{equals() and hashCode()}\label{sec:EqualsAndHashCode}
Overriding the method \lstinline|java.lang.Object::equals|\autocite{ORACLE_DOC_OBJECT_CLASS:equals} always requires to override the method \lstinline|java.lang.Object::hashCode|\autocite{ORACLE_DOC_OBJECT_CLASS:hashCode}, too – and vice versa.

\textit{This is not optional!}

The method \lstinline|equals()| returns \lstinline|true| if the given reference to refers to an object to is equals to the current one, according to \textit{your criteria what “being equal” means} in this context, and – obviously – \lstinline| false| otherwise.

So two objects can be considered to be equal when they both have the same unique id, or you require for equality that all attributes do have the same values (are also equal), or something in between. The default implementation of \lstinline|java.lang.Object.equals()| returns \lstinline|true| only if the two objects are identical\footnote{This means that both objects are the \textit{same}; the given reference points to the current object itself.}.

If two objects are equal, the result of \lstinline|hashCode()| has to be the same for both objects, but that the hash values for two objects are the same does not necessarily imply that the two objects are equal.

An implementation for the two methods should look like this:
\begin{lstlisting}[numbers=left,caption={Methods equals() and hashCode()}]
public class MyClass
{
    /**
     *  {@inheritDoc}
     */
    public boolean equals( final Object o )
    {
        var retValue = o == this;
        if( !retValue && o instanceof MyClass other
            && getClass().equals( other.getClass() ) )
        {
            retValue = Objects.equals( <attribute>, other.<attribute> )
                && Objects.equals( < … >, other.< … > );
        }
            
        //---* Done *------------------------------------------------
        return retValue;
    }   //  equals()
    
    /**
     *  {@inheritDoc}
     */
    public int hashCode()
    {
        final var retValue = Objects.hash( <attribute>, < … > );
        
        //---* Done *------------------------------------------------
        return retValue;
    }   //  hashCode()    
}
//  class MyClass
\end{lstlisting}
The check in line~10 can be omitted if \lstinline|MyClass| is \lstinline|final|. If that check is omitted for a non-\lstinline|final| class, it means that instances of derived classes can be equal to an instance of the superclass – something that is rarely wanted, especially because it would break the rule that any implementation of \lstinline|equals()| has to guarantee that
\begin{lstlisting}
a.equals( b ) == b.equals( a )
\end{lstlisting}
is always valid.

The attributes that are compared in the lines~12 and following have all to be used in \lstinline|hashCode()| to calculate the hash value.

When both a superclass and its derived classes implement \lstinline|java.lang.Object::equals| and \lstinline|java.lang.Object::hashCode|, the implementation of the derived class may call the superclass implementations of \lstinline|equals()| and \lstinline|hashCode()|:
\begin{lstlisting}[numbers=left]
public class OtherClass extends MyClass
{
    /**
     *  {@inheritDoc}
     */
    public boolean equals( final Object o )
    {
        var retValue = o == this;
        if( !retValue && o instanceof OtherClass other
            && super.equals( other ) )
        {
            retValue = Objects.equals( <attribute>, other.<attribute> )
                && Objects.equals( < … >, other.< … > );
        }
            
        //---* Done *------------------------------------------------
        return retValue;
    }   //  equals()
    
    /**
     *  {@inheritDoc}
     */
    public int hashCode()
    {
        final var retValue = Objects.hash( Integer.valueOf( super.hashCode ), <attribute>, < … > );
        
        //---* Done *------------------------------------------------
        return retValue;
    }   //  hashCode()    
}
//  class OtherClass
\end{lstlisting}
Obviously, both methods consider only the attributes that comes with the definition of the derived class; the attributes of the superclass are already covered.

\subsection{toString()}\label{sec:ToString}
According to \autocite{ORACLE_DOC_OBJECT_CLASS:toString}, the method \lstinline|toString()|
\begin{quotation}
“Returns a string representation of the object.

In general, the \lstinline|toString()| method returns a string that ‘textually represents’ this object. The result should be a concise but informative representation that is easy for a person to read. It is recommended that all subclasses override this method. […]”
\end{quotation}

What does “textually represents” mean?

The implementation for \lstinline|java.lang.Object::toString|~…
\begin{quotation}
“[…] returns a string consisting of the name of the class of which the object is an instance, the at-sign character ‘@’, and the unsigned hexadecimal representation of the hash code of the object. In other words, this method returns a string equal to the value of:\
\lstinline|getClass().getName() + '@' + Integer.toHexString( hashCode() )|”
\end{quotation}
But for an instance of \lstinline|java.lang.Integer|\autocite{ORACLE_DOC_INTEGER_CLASS}, that ‘string representation’ is just a string containing the digits for the numerical value of that object, and for an instance of \lstinline|java.lang.StringBuilder|\autocite{ORACLE_DOC_STRINGBUILDER_CLASS}, it is the current contents of the buffer.

For the class \lstinline|java.util.StringJoiner|\autocite{ORACLE_DOC_STRINGJOINER_CLASS}, \lstinline|toString()| is even the method that provides the result.

Originally, the textual representation of an object as provided by the \lstinline|toString()| method was meant only for debugging purposes, but soon it was also used for the conversion of the object's value to a string, as you can see for classes like \lstinline|java.lang.Integer|, \lstinline|java.util.UUID| or \lstinline|java.time.Instant|.

So how to implement the method \lstinline|toString()| for your method?

If your class represents objects that can be easily written as a string, you should implement \lstinline|toString()| accordingly:
\begin{lstlisting}
public final class PhoneNumber
{
    private final int m_AreaCode;
    private final int m_CountryCode;
    private final int m_SubscriberNumber;
    
    …
    
    /**
     *  {@inheritDoc}
     */
    @Override 
    public final String toString()
    {
        final var retValue = "+%d %d %d".formatted( m_CountryCode, m_AreaCode, m_SubscriberNumber );
        
        //---* Done *------------------------------------------------
        return retValue;
    }   //  toString()
}   
//  class PhoneNumber
\end{lstlisting}

Some samples for this from the Java Runtime Library are the classes below:
\begin{itemize}
\item\lstinline|java.lang.StringBuilder|\autocite{ORACLE_DOC_STRINGBUILDER_CLASS}
\item\lstinline|java.lang.StringBuffer|\autocite{ORACLE_DOC_STRINGBUFFER_CLASS}
\item\lstinline|java.util.UUID|\autocite{ORACLE_DOC_UUID_CLASS}
\item\lstinline|java.time.Instant|\autocite{ORACLE_DOC_INSTANT_CLASS} and the other classes representing time/date values from the \lstinline|java.time| package\autocite{ORACLE_DOC_TIME_PACKAGE}
\item\lstinline|java.io.File|\autocite{ORACLE_DOC_FILE_CLASS}
\item\lstinline|java.lang.Integer|\autocite{ORACLE_DOC_INTEGER_CLASS} and the other wrapper classes for the primitives
\end{itemize}

In all these cases, you can use a call to \lstinline|toString()| to embed the value of the instance into a regular text:
\begin{lstlisting}
final var phoneNumber = new PhoneNumber( … );
out.printf( "The customer's phonenumber is %s.", phoneNumber.toString() );
\end{lstlisting}

If your class is more complex and/or an output makes only sense for debugging purposes or requires additional formatting instructions, you should consider a different implementation of \lstinline|toString()|:
\begin{lstlisting}
public final class EmailMessage
{
    private final String m_Body;
    private final Map<RecipientType,List<EmailAddress>> m_Recipients;
    private final EmailAddress m_Sender;
    private final ZonedDateTime m_SentWhen;
    private final String m_Subject;
    
    …
    
    /**
     *  {@inheritDoc}
     */
    @Override 
    public final String toString()
    {
        final var buffer = new StringJoiner( ", ", "%s[".formatted( getClass().getName(), "]" )
            .add( "Body='%s'".formatted( m_Body ) )
            .add( "Recipients=%s".formatted( Objects.toString( m_Recipients ) ) )
            .add( "Sender='%s'".formatted( Objects.toString( m_EmailAddress ) ) )
            .add( "Sent_when=%s".formatted( Objects.toString( m_SentWhen ) ) )
            .add( "Subject='%s'".formatted( m_Subject ) );
            
        //---* Compose the return value *----------------------------    
        final var retValue = buffer.toString();
        
        //---* Done *------------------------------------------------
        return retValue;
    }   //  toString()
}
//  class EmailMessage
\end{lstlisting}

This is basically how the IDEs will generate the \lstinline|toString()| method. The output may look like this\footnote{The backslash indicates where I inserted a linebreak so that it looks fine in this document; otherwise it would be just one long line.}:
\begin{verbatim}
org.tquadrat.sample.EmailMessage[Body='This is the body of the\
email', Recipients=[a.b@c.de], Sender='thomas.thrien@tquadrat.\
org', Sent_when=2022-11-26T20:31:17.884636950+01:00[Europe/Ber\
lin], Subject='Ping!']
\end{verbatim}

This will work fine for a debug log, but to get it ‘pretty printed’, you may have to provide another method. Refer to chapter \tqvref{sec:FormattableInterface} for how
this could look like.

If your class is not \lstinline|final|, the method \lstinline|toString()| should not be \lstinline|final| as well. 

\subsection{clone()}\label{sec:Clone}
Per \autocite{ORACLE_DOC_OBJECT_CLASS:clone}, the method \lstinline|java.lang.Object::clone| returns a copy of the current object.

Usually, this method is \lstinline|protected| and the default implementation throws a \lstinline|java.lang.CloneNotSupportedException|\autocite{ORACLE_DOC_CLONENOTSUPPORTEDEXCEPITON_CLASS} when called.

When instances of your class should support to be cloned, it first has to implement the interface \lstinline|java.lang.Cloneable|\autocite{ORACLE_DOC_CLONEABLE_INTERFACE}, and then you need to override the method \lstinline|clone()|.

The simpliest implementation looks like this:
\begin{lstlisting}[numbers=left,caption={A simple clone() Method}]
public final MyClass implements Cloneable
{
    /**
     *  {@inheritDoc}
     */
    @Override
    public final MyClass clone()
    {
        final MyClass retValue;
        try
        {
            retValue = (MyClass) super.clone();
        }    
        catch( final CloneNotSupportedException e )
        {
            throw new UnexpectedExceptionError( e );
        }
        
        //---* Done *------------------------------------------------
        return retValue;
    }   //  clone() 
}
//  class MyClass
\end{lstlisting}
This works despite the fact that the class \lstinline|java.lang.Object| \textit{does not} implement \lstinline|java.lang.Cloneable|!

But it fails in case your class extends a class (other than \lstinline|java.lang.Object|) that does not implement \lstinline|java.lang.Cloneable|.

The implementation of \lstinline|java.lang.Object::clone| uses native code to make a shallow copy of the current object. If all attributes of the class are either immutable or primitives, the implementation shown above is sufficient.

But if any of the attributes are collections, arrays or mutable types, there is some
more work to do; in this case, your implementation of \lstinline|clone()| should declare the \lstinline|java.lang.CloneNotSupportedException|.

Such an extended implementation of \lstinline|clone()| may look like below; it is assumed that \lstinline|T| is not immutable\footnote{If \lstinline|T| will not implement  \lstinline|java.lang.Cloneable|, we can reduce the implementation of \lstinline|clone()| to just throw \lstinline|java.lang.CloneNotSupportedException|. But even if \lstinline|T| implements \lstinline|java.lang.Cloneable|, \lstinline|T::clone| can still throw that exception.}.
\begin{lstlisting}[numbers=left,caption={An extended clone() Method}]
public final MyClass<T extends Cloneable> implements Cloneable
{
    private final T [] m_Array;
    private final Collection<T> m_Collection;
    private T m_Mutable; // MAY NOT BE FINAL!!
    
    /**
     *  {@inheritDoc}
     */
    @Override
    public final MyClass clone() throws CloneNotSupportedException
    {
        final var retValue = (MyClass) super.clone();
        for( var i = 0; i < m_Array.size; ++i )
        {
            retValue.m_Array [i] = nonNull( m_Array [i] ) 
                ? m_Array [i].clone() 
                : null;
        }
        retValue.m_Collection.clear();
        for( final var t : m_Collection ) retValue.m_Collection.add( t.clone() );
        retValue.m_Mutable = m_Mutable.clone();
        
        //---* Done *------------------------------------------------
        return retValue;
    }   //  clone() 
}
//  class MyClass
\end{lstlisting}

The code in line~21 supposes that the collection does not contain any \lstinline|null| values. And obviously, the collection has to be mutable, otherwise already the attempt to clear it in line~20 would throw an exception.

Unfortunately, the interface \lstinline|java.lang.Cloneable| does not declare the method \lstinline|clone()|, therefore the code below will not work:
\begin{lstlisting}
// WILL NOT WORK!!
CloneLoop: for( final var t : m_Collection ) 
{
    if( isNull( t ) ) continue CloneLoop;
    if( t instanceof Cloneable c )
    {
        retValue.m_Collection.add( c.clone() );
    }
    else
    {
        throw new CloneNotSupportedException( t.toString() );
    }    
}   //  CloneLoop:
\end{lstlisting}

So in case a collection (or an array) may contain components that can or cannot be cloned, the test is a bit more complex:
\begin{lstlisting}
CloneLoop: for( final var t : m_Collection ) 
{
    if( isNull( t ) ) continue CloneLoop;
    boolean isCloneable = false;
    if( t instanceof Cloneable )
    {
    	isCloneable = stream( getClass().getMethods() )
    	    .filter( m -> m.getParameterCount() == 0 )
    	    .map( Method::getName )
    	    .filter( n -> n.equals( "clone" ) )
    	    .count() == 1;
    }
    
    if( isCloneable )
    {
        retValue.m_Collection.add( c.clone() );
    }
    else
    {
        throw new CloneNotSupportedException( t.toString() );
    }    
}   //  CloneLoop:
\end{lstlisting}
Instead of the stream operation, you can also use \lstinline|java.lang.Class::getMethod| and respond to the \lstinline|java.lang.NoSuchMethodException| for the check.

If you have to implement the method \lstinline|java.lang.Object::clone| for lots of classes, it could make sense to implement your own \lstinline|Cloneable| interface that then can be used with the \lstinline|instanceof| operator and the pattern:
\begin{lstlisting}
public interface Cloneable<T> extends java.lang.Cloneable
{
    /**
     *  Forces the implementation of
     *  {@link Object#clone()}
     *
     *  @returns The cloned object.
     *  @throws  CloneNotSupportedException One of the mutable
     *      attributes is not cloneable.
     */
    public T clone() throws CloneNotSupportedException; 
}
//  interface Cloneable
\end{lstlisting}

There are several ongoing discussions whether the API that was defined through the \lstinline|java.lang.Object::clone| is generally usefull or more a pain in the ass, and as far as I am aware, these discussions will last for some more time.

I am not a friend of \lstinline|clone()| and I try to avoid its implementation whenever possible. This means that sometimes it is not possible to circumvent the implementation of that method.

My recommendation is to ignore the method \lstinline|java.lang.Object::clone| unless there is a strict requirement to use it.
